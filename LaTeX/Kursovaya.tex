\documentclass[xcolor=table]{beamer}
\mode<presentation>
{
  \usetheme{Madrid}      
  \usecolortheme{default} 
  \usefonttheme{default}  
  \setbeamertemplate{navigation symbols}{}
  \setbeamertemplate{caption}[numbered]
} 

\usepackage[english, russian]{babel}
\usepackage[utf8x]{inputenc}

\title{Модели экономической динамики и их устойчивость}
\author{Курсовая работа Гончаровой Алины}
\institute{Научный руководитель: доцент, кандидат ф.-м. наук, Ефимова Г.А.}
\date{Одесса - 2018}

\begin{document}
\begin{frame}
  \titlepage
\end{frame}
\section{Цель работы}
\begin{frame}{Цель работы}
\begin{itemize}
  \item Рассмотрение паутинообразной модели, модели Харрода-Домара и модели Солоу.
  \item Изучение устойчивости рассмотренных.
  
\end{itemize}
\end{frame}


\section{Паутинообразная модель}
\begin{frame}{Паутинообразная модель}
\begin{itemize}
\item[] $Q_{S}(t) = S_{t}(p_{t-1})$ - спрос,
\item[] $Q_{D}(t) = D_{t}(p_{t})  $ - предложение,
\item[] $Q_{t}^{S}=S_{t}(p_{t-1}), Q_{t}^{D}=S_{t}(p_{t-1}).$
\end{itemize}


$$ D_{t}(p_{t})=S_{t}(p_{t-1}) $$
\end{frame}


\section{Задача 1}
\begin{frame}{Задача 1}
\hspace{0.7cm} Имеется паутинообразная модель
$$S_{t} = 20 + 30p_{t-1}$$
$$D_{t} = 100 - 50p_{t} $$
\hspace{0.7cm} Пусть $p_{0}=0,5$, чему равно $p_{1}$?

\hspace{0.7cm} Т.к. $D_{t} = S_{t}$ ,имеем равенство:
$$20 + 30p_{t-1} = 100 - 50p_{t} $$

\hspace{0.7cm} Подставляя имеющиеся значения получаем:
$$ p_{1} = 1,3, p_{2} = 0,8, p_{3} = 1,1, p_{4} = 0,9, \dots $$

\hspace{0.7cm}Найдем значения спроса и предложения:
$$S_{1}(p_{0}) = D_{1}(p_{1}) = 35, S_{2}(p_{1}) = D_{2}(p_{2}) = 59, S_{3}(p_{2}) = D_{3}(p_{3}) = 44,6, \dots$$
\end{frame}

\subsection{Задача 2}
\begin{frame}{Задача 2}
\hspace{0.7cm} Пусть в паутинообразной модели функция спроса равна $D_{t} = \frac{3}{p_{t}}$, функция предложения $S_{t}=5p_{t-1}$, $p_{0}=1$.
$$ D_{t} = S_{t} $$

$$p_{t} =  \frac{3}{5p_{t-1}}$$

\hspace{0.7cm} Следовательно, $$p_{1}=\frac{3}{5},p_{2}=1,p_{3}=\frac{3}{5},p_{4} = 1, \dots$$
$$S_{1}=D_{1}=5, S_{2}=D_{2} = 3, S_{2}=D_{2} = 5 \dots$$

\end{frame}

\frame{
    \frametitle{Конец}
    \centering
    {
     Спасибо за внимание!
    }
}
\end{document}
