\documentclass[a4paper, fontsize=14pt, oneside]{scrartcl}
\usepackage[T2A]{fontenc}
\usepackage[utf8x]{inputenc}
%\usepackage[russian]{babel}
\usepackage[ukrainian]{babel}
\usepackage[a4paper, top=20mm, left=30mm, right=15mm, bottom=20mm]{geometry}
\usepackage{misccorr}  
\usepackage{indentfirst}
\usepackage{}
\setlength{\parindent}{1.5cm}
\setlength{\parskip}{0.2cm}
\usepackage{epigraph}
\setlength\epigraphrule{0pt}
\setlength\epigraphwidth{.57\textwidth}
\pagestyle{empty}
\usepackage{ragged2e}

\begin{document}
	\begin{center}
		\Large{\textbf{2. Розрахунок показників динаміки розвитку економічних процесів}}
		\end{center}
        
        Після виявлення і усунення аномальних спостережень у часовому ряді, необхідно виявити закономірності динаміки досліджуваних явищ. Закономірності можуть бути виявлені за допомогою аналітичних показників, тобто аналізу абсолютної швидкості і інтесивності розвитку явищ.
        
        Для дослідження часового ряду необхідно, щоб ріні ряду були однорідні. Причиною неоднорідності даних можуть бути дані різних регіонів, об'єднань, різних років, різних масштабів, різних структур сукупності і інші.
        
        При дослідженні часових рядів розраховують абсолютний приріст:
        \[ \Delta y_{t} = y_{t} - y_{i-k}, i = 2, 3, \dots, n, k = 1, 2, \dots, n-1  - \]
       визначає початковий рівень і може бути різним. Якщо $ k = 1 $ , то отримаємо ланцюгові показники, а при $ k = i - 1 $ маємо базисні показники з початковим рівнем ряду в якості базисного.
       
       Отже, абсолютний приріст є величина зміни показника за час між працювальними періодами.
       
       Середнім абсолютним приростом, тобто швидкість зміни абсолютного приросту, називають величину \[ \Delta \bar{y_{i}} =\frac{y_{i} - y_{i-k}}{k}.\] 
       
       Якщо $ i = n, k = n - 1 $, то отримаємо середній абсолютний приріст за весь період спостережень $ \Delta \bar{y} =\frac{y_{n} - y_{1}}{n - 1} $, що характеризує середню швидкість зміни часового ряду.
       
       Для визначення відносної швидкості зміни явища за одиницю часу використовують відносні показники.
       
       Коєфіцієнт зростання для $ i $-го періоду розраховується за формулою \[ K_{i(3P)}=\frac{y_{i}}{y_{i-k}}.\]
       
       Якщо $ K_{i(np)} > 1 $, то рівень зростає, а при $ K_{i(3P)} < 1 - $ зменшується, а при  $ K_{i(3P)} = 1 $ рівень не змінюється.
       Коєфіцієнт приросту розраховується так: \[ K_{i(np)} = K_{i(3P)} -1 = \frac{y_{i}-y_{i-k}}{y_{i-k}} \]
       Також використовують показники, на скільки відсотків рівень одного періоду змінився в порінянні з рівнем іншого періоду.
       
       Може статися так, що зменшення темпу приросту не супроводжується зменшенням абсолютних приростів.
       
       Середній темп зростання характеризує середню швидкість зміни явища за весь період: \[ \bar{T}_{(3P)} = \sqrt[n-1]{\frac{y_{n}}{y_{1}}} 100\%. \]
       
       Якщо ряд має сильні коливання, то ця формула дає не дуже точні результати і тому краще користуватися іншою: \[ \bar{T}_{(3P)} = \sqrt[n-1]{\frac{\widehat{y_{n}}}{\widehat{y_{1}}}} 100\%. \], де $\widehat{y_{n}}$ і $\widehat{y_{1}}$ - згладжені за рівнянням тренду початковий і кінцевий рівні часового ряду.
       Ще використовується такий показник, як середній рівень ряду: $ \bar{y} = \frac{\sum\limits_{i=1}^{n} y_{i}}{n} $
       

\newpage

       \begin{center}
		\Large{\textbf{3. Автокореляція рівнів часового ряду}}
		\end{center}
        
        Одна з головних відмінностей послідовності спостережень часового ряду від випадкової вибірки полягає в тому, що члени часового ряду є статистично взаємозалежними. Кореляційну залежність між послідовними рівнями часового ряду називають автокореяцією рівнів ряду. Вона визначається за допомогою лінійного коєфіцієнта кореляції між рівнями цього ж ряду, зсунутими на декілька кроків в часу за формулою \[ r_{k} = \frac{(n-k)\sum\limits_{t=1}^{n-k}y_{t}y_{t+k} - \sum\limits_{t=1}^{n-k}y_{t} \sum\limits_{t=1}^{n-k}y_{t+k}}{\sqrt{[(n-k) \sum\limits_{t=1}^{n-k}y_{t}^{2} - (\sum\limits_{t=1}^{n-k}y_{t})^{2}][(n-k)\sum\limits_{t=1}^{n-k}y_{t+k}^{2} - (\sum\limits_{t=1}^{n-k}y_{t+k})^{2}]}} \]
        
        Порядок коэфіцієнта автокореляції визначається параметром $k$. Цей параметр називають лагом. Зі збільшенням лагу кількість пар значень, за якими розраховується коєфіцієнт автокореляції зменшується. Вважається, що максимальний лаг можна обирати не більше чверті довжини часового ряду. 
        
        Послідовність коефіцієнтів автокореляції назівається кореляційною функцією часового ряду. Графік досяжності її значень від величини лагу (порядка коефіцієнта автокореляції) називають корелограмою.
        
        Аналіз автокореляційної функції і корелограми дозволяє виявити структуру ряду, а саме:
        \begin{enumerate}
			\item  Випадковий ряд. Якщо часовий ряд є повністю випадковим, тоді для великих $k, r_{k} = 0 $. Для випадкових часових рядів $\left\{ r_{k} \right \} \in N(0, \frac{1}{n})$.
		\end{enumerate}
        
\end{document}



